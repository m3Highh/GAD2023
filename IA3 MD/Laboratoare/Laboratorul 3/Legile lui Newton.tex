\documentclass[11pt]{article}
	
	\usepackage{adjustbox}
	\usepackage{amsmath}
	\usepackage{amssymb}
	\usepackage{float}
	\usepackage[T1]{fontenc}
	\usepackage{graphicx}
	\usepackage[utf8]{inputenc}
	\usepackage{mathtools}
	\usepackage{multicol}
	\usepackage{multirow}
	\usepackage{wasysym}
	\usepackage[paperheight=29.7cm,paperwidth=21.0cm,left=2.54cm,right=2.54cm,top=2.54cm,bottom=2.54cm]{geometry}
	\usepackage[hidelinks]{hyperref}
	
	
	\setlength\parindent{0pt}
	\renewcommand{\arraystretch}{1.3}
	
	
	
	\begin{document}
	
	\begin{center}
	{\large \textbf{Legile lui Newton}}
	\end{center}
	
	
	\begin{center}
	Pleantă Mihai-Alexandru
	\end{center}
	
	
	{\footnotesize \textbf{Rezumat}. In lucrare sunt prezentate elemente introductive privind legile lui Newton}
	
	\begin{enumerate}
		\item \textbf{Introducere}
	
	Legile lui Newton (sau principiile fundamentale ale mecanicii) sunt trei legi ale fizicii care dau o relație directă între forțele care acționează asupra unui \href{https://ro.wikipedia.org/w/index.php?title=Corp_fizic&action=edit&redlink=1}{corp} și \href{https://ro.wikipedia.org/wiki/Mi\%C8\%99care}{mișcarea} acelui corp. Ele au fost enunțate de \href{https://ro.wikipedia.org/wiki/Isaac_Newton}{Sir Isaac Newton} (bazat și pe studiile lui \href{https://ro.wikipedia.org/wiki/Galileo_Galilei}{Galilei}) în lucrarea sa \href{https://ro.wikipedia.org/wiki/Philosophiae_Naturalis_Principia_Mathematica}{Philosophiae Naturalis Principia Mathematica} (\href{https://ro.wikipedia.org/wiki/1687}{1687}). Aceste legi formează baza \href{https://ro.wikipedia.org/wiki/Mecanica_clasic\%C4\%83}{mecanicii clasice}.
	
	\end{enumerate}
	\vspace{1\baselineskip}
	\begin{enumerate}
		\item \textbf{Principiul I și al II-lea al mecanicii}
	
	Principiul I al mecanicii sau principiul inerției a fost formulat pentru prima dată de Galilei și este cunoscut sub forma: „Orice corp își menține starea de repaus sau de mișcare rectilinie uniformă atât timp cât asupra sa nu acționează alte forțe sau suma forțelor care acționează asupra sa este nulă."
	
	Deoarece mișcarea este caracterizată în raport cu un \href{https://ro.wikipedia.org/wiki/Sistem_de_referin\%C8\%9B\%C4\%83}{sistem de referință} ales arbitrar, mișcarea are caracter relativ. În acest sens, Galilei a formulat principiul relativității mișcării mecanice. Să considerăm un călător aflat într-un \href{https://ro.wikipedia.org/wiki/Vehicul}{vehicul} care se deplasează rectiliniu și uniform. Călătorul se poate găsi într-una din stările:
	
	\begin{itemize}
		\item în repaus, în raport cu sistemul de referință legat de vehicul;
	
		\item în mișcare rectilinie uniformă cu o viteză egală cu cea a vehiculului față de un sistem de referință legat de \href{https://ro.wikipedia.org/wiki/P\%C4\%83m\%C3\%A2nt}{Pământ};
	
		\item în mișcare accelerată, în raport cu un sistem de referință legat de \href{https://ro.wikipedia.org/wiki/Soare}{Soare}, deoarece Pământul este în mișcare accelerată față de Soare.
	
	\end{itemize}
	Toate \href{https://ro.wikipedia.org/wiki/Sistem_de_referin\%C8\%9B\%C4\%83}{sistemele de referință} ce se mișcă rectiliniu uniform se numesc \href{https://ro.wikipedia.org/wiki/Sistem_de_referin\%C8\%9B\%C4\%83_iner\%C8\%9Bial}{sisteme de referință inerțiale}. În aceste sisteme de referință este valabil principiul inerției.
	
	Principiul al II-lea al mecanicii: Newton a descoperit faptul că o forță care acționează asupra unui corp îi imprimă acestuia o accelerație, proporțională cu forța și invers proporțională cu masa corpului:
	
	\begin{table}[H]
	\begin{adjustbox}{max width=\textwidth}
	\begin{tabular}{p{15.0cm}p{1.0cm}}
	\multicolumn{1}{p{15.0cm}}{\(\overrightarrow{F} = m\overrightarrow{a}\) \newline
	\end{enumerate}
	\centering
	(principiul forței sau legea a doua a dinamici)} & 
	\multicolumn{1}{p{1.0cm}}{\raggedleft
	(1)} \\ 
	\end{tabular}
	\end{adjustbox}
	\end{table}
	\vspace{1\baselineskip}
	\textbf{Tabelul 1}. Unitățile de măsură a componentelor principiului forței
	
	\begin{table}[H]
	\begin{adjustbox}{max width=\textwidth}
	\begin{tabular}{p{2.0cm}p{7.0cm}p{7.0cm}}
	\hline
	\multicolumn{1}{|p{2.0cm}}{\textit{Nr.crt}} & 
	\multicolumn{1}{|p{7.0cm}}{\textit{Mărime}} & 
	\multicolumn{1}{|p{7.0cm}|}{\textit{Unitate de măsură}} \\ 
	\hline
	\multicolumn{1}{|p{2.0cm}}{1} & 
	\multicolumn{1}{|p{7.0cm}}{masa} & 
	\multicolumn{1}{|p{7.0cm}|}{[kg]} \\ 
	\hline
	\multicolumn{1}{|p{2.0cm}}{2} & 
	\multicolumn{1}{|p{7.0cm}}{accelerația} & 
	\multicolumn{1}{|p{7.0cm}|}{[m/s\textsuperscript{2}]} \\ 
	\hline
	\multicolumn{1}{|p{2.0cm}}{3} & 
	\multicolumn{1}{|p{7.0cm}}{forța} & 
	\multicolumn{1}{|p{7.0cm}|}{[(kg{\scriptsize $\bullet$}m)/s\textsuperscript{2}]} \\ 
	\hline
	\end{tabular}
	\end{adjustbox}
	\end{table}
	\begin{table}[H]
	\begin{adjustbox}{max width=\textwidth}
	\begin{tabular}{p{15.9cm}}
	\multicolumn{1}{p{15.9cm}}{} \\ 
	\multicolumn{1}{p{15.9cm}}{\centering
	\textbf{Figura 1.} Dependența forței de tracțiune față de accelerație. \newline
	\centering
	\includegraphics[width=9.25cm,height=6.96cm]{./images/image1.png}
	} \\ 
	\end{tabular}
	\end{adjustbox}
	\end{table}
	\vspace{4\baselineskip}
	\begin{enumerate}
		\item \textbf{Concluzii}
	
	\end{enumerate}
	În concluzie legile lui Newton stau la baza mecanicii clasice.
	
	\textbf{Bibliografie}
	
	1 Wikipedia-Legile lui Newton: https://ro.wikipedia.org/wiki/Legile\_lui\_Newton
	
	
	\end{document}